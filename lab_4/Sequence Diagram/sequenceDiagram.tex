\documentclass[a4paper,12pt]{article}

\usepackage{geometry}
\geometry{margin=2cm}
\usepackage{fontspec}
\setmainfont{Times New Roman}
\usepackage{enumitem}
\usepackage{titlesec}

\titleformat{\section}{\normalfont\Large\bfseries}{\thesection.}{1em}{}

\begin{document}

\begin{center}
    \Large\textbf{Пояснительный текст к диаграмме последовательности}
\end{center}

\section{Общее описание}
Диаграмма последовательности отражает пошаговый обмен сообщениями между участниками процесса выдачи книги в системе управления библиотекой.  
В диаграмме участвуют внешнее действующее лицо (\textit{Читатель}), пользовательский интерфейс, основной модуль системы, а также отдельные менеджеры, отвечающие за обработку выдач, работу с каталогом, начисление штрафов и взаимодействие с базой данных.

\section{Основной сценарий}
Последовательность начинается с запроса читателя на выдачу книги.  
Пользовательский интерфейс передаёт запрос в ядро системы, после чего \textit{Loan Manager} инициирует оформление выдачи. Далее выполняется проверка доступности книги и запись факта выдачи в базу данных.

Если книга доступна, происходит оформление выдачи, возможное начисление штрафа (опциональный блок \texttt{opt}), после чего пользователю возвращается подтверждение.

Если книга недоступна, система использует альтернативную ветку (\texttt{alt}) и уведомляет читателя об отказе.

\section{Заключение}
Диаграмма показывает динамическое взаимодействие компонентов системы при выполнении операции выдачи книги.  
Она демонстрирует ветвление логики, обращение к вспомогательным сервисам и последовательность сообщений, необходимых для успешной или неуспешной обработки запроса.

\end{document}

