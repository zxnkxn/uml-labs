\documentclass[a4paper,12pt]{article}

\usepackage{geometry}
\geometry{margin=2cm}
\usepackage{fontspec}
\setmainfont{Times New Roman}
\usepackage{enumitem}
\usepackage{titlesec}
\titleformat{\section}{\normalfont\Large\bfseries}{\thesection.}{1em}{}
\titleformat{\subsection}{\normalfont\large\bfseries}{\thesubsection.}{0.5em}{}

\title{Пояснение к диаграмме классов системы управления библиотекой}
\date{}

\begin{document}
\maketitle

\section{Описание классов}

\subsection{Человек}
Класс \textbf{Человек} — абстрактный базовый класс, содержащий общие атрибуты всех пользователей системы:
\begin{itemize}
    \item \textbf{ФИО} --- строка с именем пользователя;
    \item \textbf{Дата рождения} --- дата рождения пользователя.
\end{itemize}
Этот класс служит основой для специализированных классов пользователей.

\subsection{Читатель}
Класс \textbf{Читатель} наследует свойства класса \textbf{Человек} и содержит дополнительные атрибуты, характерные для конечного пользователя библиотеки (например, статус задолженности).  
\textbf{Связи:} ассоциации с классами \textbf{Книга} (через класс ассоциации \textbf{Бронирование}) и с классом \textbf{Выдача книги}.

\subsection{Библиотекарь}
Класс \textbf{Библиотекарь} наследует класс \textbf{Человек} и содержит служебные атрибуты (например, список обработанных выдач).  
\textbf{Связи:} ассоциация с классом \textbf{Выдача книги} (библиотекарь оформляет выдачи) и с классом \textbf{Отчёт} (библиотекарь может создавать отчёты).

\subsection{Администратор}
Класс \textbf{Администратор} наследует класс \textbf{Библиотекарь} (и, косвенно, \textbf{Человек}). Такое наследование выбрано целенаправленно: администратор обладает всеми возможностями библиотекаря и дополнительно — правами просмотра отчётов.  
\textbf{Связи:} ассоциация с классом \textbf{Отчёт} (просмотр отчёта).

\subsection{Книга}
Класс \textbf{Книга} включает свойства: ISBN, название, автор, год издания, статус (доступна/выдана/забронирована).  
\textbf{Связи:}
\begin{itemize}
    \item с \textbf{Выдача книги} — отношение композиции (выдача привязана к конкретной книге);
    \item с \textbf{Читатель} — прямая ассоциация;
    \item с \textbf{Бронирование} — используется класс ассоциации, бронирование связывает конкретную книгу и читателя и хранит атрибуты бронирования.
\end{itemize}

\subsection{Бронирование}
Класс \textbf{Бронирование} описывает факт резервирования книги читателем и содержит свои атрибуты: дата бронирования, срок действия.  
\textbf{Связи:} класс ассоциации между \textbf{Читателем} и \textbf{Книгой}. Именно через этот класс удобно хранить данные, относящиеся к самому акту бронирования.

\subsection{Выдача книги}
Класс \textbf{Выдача книги} фиксирует факт выдачи экземпляра книги конкретному читателю и включает атрибуты: дата выдачи, дата возврата.  
\textbf{Связи:}
\begin{itemize}
    \item с \textbf{Книгой} — композиция: каждая выдача относится к конкретной книге, и без связи с книгой запись выдачи теряет смысл;
    \item с \textbf{Читателем} — ассоциация: выдача оформляется на конкретного читателя;
    \item с \textbf{Библиотекарем} — ассоциация: библиотекарь оформляет выдачу;
    \item с \textbf{Штрафом} — ассоциация: штрафы начисляются в рамках конкретной выдачи (например, за просрочку). Штрафы логически связаны с выдачей и хранятся в отдельном классе.
\end{itemize}

\subsection{Штраф}
Класс \textbf{Штраф} содержит атрибуты: сумма, дата начисления.  
\textbf{Связь:} ассоциация с классом \textbf{Выдача книги}. Штраф относится к конкретной выдаче (за конкретное нарушение), поэтому связь между ними — прямая ассоциация.

\subsection{Отчёт}
Класс \textbf{Отчёт} содержит информацию, сгенерированную по результатам работы системы (по книгам, по читателям, по работе библиотеки).  
\textbf{Связи:}
\begin{itemize}
    \item с \textbf{Библиотекарем} — ассоциация: отчёт создаётся библиотекарем;
    \item с \textbf{Администратором} — ассоциация: администратор просматривает отчёты.
\end{itemize}

\section{Кратности}

Ниже приведено пояснение выбранных кратностей для основных отношений диаграммы.

\subsection{Читатель --- Книга}
Обе стороны имеют кратность \textbf{0..*}. Это значит, что:
\begin{itemize}
    \item у одного читателя может быть ноль или много связанных книг;
    \item у одной книги может быть ноль или много связанных читателей.
\end{itemize}

\subsection{Книга --- Выдача книги (композиция)}
У книги может быть \textbf{0..*} записей выдач, а каждая выдача относится к \textbf{1} конкретной книге. Композиция выбрана потому, что запись о выдаче теряет смысл без указания конкретной книги.

\subsection{Выдача книги --- Читатель (ассоциация)}
Каждая выдача оформляется на \textbf{1} читателя, а один читатель может иметь \textbf{0..*} выдач (активных и завершённых).

\subsection{Выдача книги --- Библиотекарь (ассоциация)}
Каждая выдача оформляется \textbf{1} библиотекарем, а один библиотекарь может оформлять \textbf{0..*} выдач.

\subsection{Выдача книги --- Штраф (ассоциация)}
Одна выдача может иметь \textbf{0..*} штрафных записей (в зависимости от бизнес-логики: один штраф или несколько событий начисления), а каждый штраф относится к \textbf{1} конкретной выдаче. Ассоциация достаточна, так как штраф логически связан с выдачей, но его запись может храниться и обрабатываться отдельно.

\subsection{Библиотекарь --- Отчёт (ассоциация)}
Один библиотекарь может создавать \textbf{0..*} отчётов, а каждый отчёт создаётся \textbf{1} библиотекарем.

\subsection{Администратор --- Отчёт (ассоциация)}
Администратор может просматривать \textbf{0..*} отчётов; один отчёт может просматриваться \textbf{0..*} администраторами.

\section{Заключение}

В документе отдельно перечислены классы и типы их связей, после чего приведено подробное объяснение выбранных кратностей. 

\end{document}

