\documentclass[a4paper,12pt]{article}

\usepackage{geometry}
\geometry{margin=2cm}
\usepackage{fontspec}
\setmainfont{Times New Roman}
\usepackage{enumitem}
\usepackage{titlesec}

\titleformat{\section}{\normalfont\Large\bfseries}{\thesection.}{1em}{}

\begin{document}

\begin{center}
    \Large\textbf{Пояснительный текст к диаграмме деятельности}
\end{center}

\section{Общее описание}
Диаграмма деятельности описывает процесс выдачи книги читателю с участием информационной системы (ИС) библиотеки.  
Процесс отражает взаимодействие внешних участников с системой: запрос книги, проверку её статуса, проверку статуса читателя, возможные уведомления о невозможности выдачи, а также оформление выдачи книги и обновление отчётных данных.

\section{Сценарий процесса}
\begin{enumerate}[label=\arabic*.]
    \item \textbf{Вне ИС:} Читатель делает запрос книги.
    \item \textbf{ИС:} Проверяется статус книги (доступна, забронирована или выдана).
    \item \textbf{ИС:} Условный блок:
    \begin{enumerate}[label*=\arabic*.]
        \item \textbf{Вне ИС:} Если книга недоступна, отправляется уведомление о недоступности книги, после чего процесс завершается.
        \item \textbf{ИС:} Если книга доступна, проверяется статус читателя.
    \end{enumerate}
    \item \textbf{ИС:} Второй условный блок:
    \begin{enumerate}[label*=\arabic*.]
        \item \textbf{Вне ИС:} Если у читателя есть запрет на выдачу (например, задолженность или превышен лимит по книгам), отправляется уведомление о запрете выдачи, после чего процесс завершается.
        \item \textbf{ИС:} Если запрета нет, система направляет уведомление библиотекарю о возможности выдачи книги.
    \end{enumerate}
    \item \textbf{Вне ИС:} Библиотекарь вручает книгу читателю.
    \item \textbf{ИС:} Формируется отчёт о выдаче книги.
    \item \textbf{ИС:} Обновляются отчётные данные (статистика выданных книг).
\end{enumerate}

\section{Заключение}
Диаграмма отражает последовательность действий и обмена информацией между внешними участниками и информационной системой при выдаче книги.  
Результатом является успешная выдача книги и обновление статистических данных в ИС.

\end{document}
