\documentclass{article}

\usepackage[margin=2cm]{geometry}
\usepackage{fontspec}

\setmainfont{Times New Roman}

\begin{document}
	
	\section*{Описание диаграммы размещения}
	
	Диаграмма размещения состоит из трёх основных узлов (nodes):
	
	\begin{enumerate}
		\item \textbf{Client Device} — клиентское устройство.
		На нём развёрнут один артефакт:
		\begin{itemize}
			\item \textbf{UI} — пользовательский интерфейс.
		\end{itemize}
		
		\item \textbf{Application Server} — сервер приложения.
		На этом узле размещаются два артефакта:
		\begin{itemize}
			\item \textbf{CoreApp} — основной компонент приложения.
			\item \textbf{Services} — набор сервисных компонентов.
		\end{itemize}
		
		\item \textbf{Database Server} — сервер базы данных.
		Здесь развёрнут один артефакт:
		\begin{itemize}
			\item \textbf{DB} — база данных.
		\end{itemize}
	\end{enumerate}
	
	Связи между узлами:
	
	\begin{itemize}
		\item Между \textbf{Client Device} и \textbf{Application Server} установлена связь с кратностью 
		от \textbf{многих к одному}.
		
		\item Между \textbf{Application Server} и \textbf{Database Server} указана связь 
		\textbf{один к одному}, что отражает подключение серверного приложения к одному экземпляру базы данных.
	\end{itemize}
	
	Таким образом, диаграмма отображает развёртывание основных компонентов системы: пользовательский интерфейс выполняется на клиентском устройстве, логика приложения — на сервере, а хранение данных осуществляется на выделенном сервере базы данных.
	
\end{document}
