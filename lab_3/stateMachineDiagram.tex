\documentclass[a4paper,12pt]{article}

\usepackage{geometry}
\geometry{margin=2cm}
\usepackage{fontspec}
\setmainfont{Times New Roman}
\usepackage{enumitem}
\usepackage{titlesec}
\titleformat{\section}{\normalfont\Large\bfseries}{\thesection.}{1em}{}

\title{Описание диаграммы состояний пользовательского интерфейса}
\author{Система управления библиотекой}
\date{}

\begin{document}

\maketitle

\section{Общее описание}

На представленной диаграмме состояний показана логика работы пользовательского интерфейса системы управления библиотекой. 
Каждое состояние отражает определённый экран приложения, а переходы между состояниями соответствуют действиям пользователя, таким как выбор вкладки или нажатие кнопки.

\section{Описание состояний}

\begin{itemize}[noitemsep]
    \item \textbf{Выданные книги} — экран, на котором отображается список книг, находящихся на руках у читателей. Указаны имя читателя, дата выдачи, срок возврата и статус книги (в срок или просрочено);
    \item \textbf{Список задолженностей} — экран, содержащий перечень пользователей с невозвращёнными книгами и суммой задолженности. Также отображается общая сумма задолженностей;
    \item \textbf{Доступные книги} — список книг, находящихся в библиотеке и доступных для выдачи. Для каждой книги указаны автор, жанр, год издания и количество экземпляров;
    \item \textbf{Отчёты} — интерфейс для библиотекаря, в котором отображаются существующие отчёты и доступна кнопка для создания нового отчёта;
    \item \textbf{Создание нового отчёта} — всплывающее окно, где библиотекарь вводит название, тип и содержание отчёта. После нажатия кнопки «Создать» происходит возврат к экрану «Отчёты».
\end{itemize}

\section{Описание переходов}

\begin{itemize}[noitemsep]
    \item Переходы между экранами «Выданные книги», «Список задолженностей», «Доступные книги» и «Отчёты» осуществляются через верхнее меню системы;
    \item Из экрана «Отчёты» пользователь может перейти в окно «Создание нового отчёта», нажав кнопку «Создать отчёт»;
    \item После завершения или отмены создания отчёта система возвращает пользователя обратно в экран «Отчёты».
\end{itemize}

\section{Работа интерфейса}

Диаграмма демонстрирует, что пользователь (библиотекарь) может свободно переключаться между основными разделами приложения, управлять данными о книгах и пользователях, а также формировать отчёты. 
Такая структура обеспечивает простоту навигации и наглядность интерфейса.

\section{Заключение}

Разработанная диаграмма состояний отображает поведение пользовательского интерфейса системы управления библиотекой. 
Она описывает последовательность экранов и переходов между ними, отражая взаимодействие пользователя с системой. 
Данный подход позволяет формализовать структуру интерфейса и повысить удобство проектирования прикладного программного обеспечения.

\end{document}
