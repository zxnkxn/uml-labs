\documentclass[a4paper,12pt]{article}

\usepackage{geometry}
\geometry{margin=2cm}
\usepackage{fontspec}
\setmainfont{Times New Roman}
\usepackage{enumitem}
\usepackage{titlesec}

\titleformat{\section}{\normalfont\Large\bfseries}{\thesection.}{1em}{}

\begin{document}

\begin{center}
    \Large\textbf{Пояснительный текст к диаграмме компонентов}
\end{center}

\section{Общее описание}
Диаграмма компонентов отражает архитектурную структуру информационной системы библиотеки.  
На диаграмме выделены основные программные компоненты, их взаимные связи и соответствующие программные артефакты.  
Диаграмма показывает, какие функциональные модули образуют систему, какие зависимости между ними существуют,
а также какие артефакты реализуют те или иные компоненты.

\section{Состав системы}
Система состоит из четырёх ключевых групп элементов:
\begin{enumerate}[label=\arabic*.]
    \item \textbf{Пользовательский интерфейс (UI)} — обеспечивает взаимодействие читателя и библиотекаря с системой.
    \item \textbf{Ядро системы (Core)} — главный посредник между интерфейсом и сервисными модулями, координирующий логику работы.
    \item \textbf{Сервисные компоненты} — набор независимых менеджеров, отвечающих за авторизацию, работу с каталогом, бронирование, выдачу, штрафы, отчёты, мониторинг и уведомления.
    \item \textbf{База данных (DB)} — центральное хранилище данных, с которым взаимодействуют сервисные компоненты.
\end{enumerate}

Каждый компонент представлен на диаграмме в виде отдельного UML-блока, для компонентов созданы артефакты: \textit{UI}, \textit{CoreApp}, \textit{Services} и \textit{DB}.  
Пунктирные линии типа \textit{manifest} обозначают связь между компонентом и артефактом, в котором он реализован.

\section{Взаимодействие компонентов}
Пользовательский интерфейс взаимодействует с ядром системы, передавая ему запросы пользователей.  
Ядро в свою очередь вызывает соответствующие сервисные компоненты, такие как менеджер каталога, бронирования или штрафов.  
Сервисные модули обращаются к базе данных для чтения или изменения информации.  
Таким образом, диаграмма отражает модульную структуру системы и показывает, как разделена ответственность между компонентами.

\section{Заключение}
Диаграмма компонентов демонстрирует архитектуру программной системы в терминах основных модулей, их реализации и связей.
Такое представление позволяет оценить структуру приложения, определить зависимости и упростить дальнейшее сопровождение и развитие системы.

\end{document}

