\documentclass[a4paper,12pt]{article}

\usepackage{geometry}
\geometry{margin=2cm}
\usepackage{fontspec}
\setmainfont{Times New Roman}
\usepackage{enumitem}
\usepackage{titlesec}
\titleformat{\section}{\normalfont\Large\bfseries}{\thesection.}{1em}{}

\title{Описание диаграммы вариантов использования}
\author{Система управления библиотекой}
\date{}

\begin{document}

\maketitle

\section{Общее описание}

На представленной диаграмме вариантов использования изображены основные действующие лица и функции системы управления библиотекой. 
Система предназначена для автоматизации процессов учёта, выдачи и возврата книг, а также для контроля состояния фонда и пользователей.

В системе выделены три основных действующих лица:
\begin{itemize}
    \item \textbf{Читатель} — конечный пользователь, который просматривает каталог книг, бронирует и продлевает срок пользования книгой;
    \item \textbf{Библиотекарь} — сотрудник, ответственный за выдачу и возврат книг, ведение каталога и контроль задолженностей;
    \item \textbf{Администратор} — пользователь с расширенными правами, осуществляющий управление системой, удаление книг и мониторинг её состояния.
\end{itemize}

\section{Читатель}

Основные варианты использования для действующего лица \textbf{Читатель}:
\begin{itemize}[noitemsep]
    \item \textbf{Просмотр каталога книг} — пользователь получает доступ к полному списку книг в библиотеке;
    \item \textbf{Добавить книгу в избранное} — позволяет сохранить интересующие книги для последующего просмотра;
    \item \textbf{Бронирование книги} — резервирование выбранного экземпляра, если книга доступна;
    \item \textbf{Продление срока пользования книгой} — возможность продлить срок возврата книги при отсутствии брони другими пользователями.
\end{itemize}

\section{Библиотекарь}

Действующее лицо \textbf{Библиотекарь} обладает следующими вариантами использования:
\begin{itemize}[noitemsep]
    \item \textbf{Выдача книги} — основной процесс обслуживания читателя:
        \begin{itemize}
            \item включает проверку наличия книги (\textit{include});
            \item расширяется сценарием «Резервирование книги», если книга недоступна (\textit{extend}).
        \end{itemize}
    \item \textbf{Возврат книги} — оформление возврата книги:
        \begin{itemize}
            \item включает проверку задолженности (\textit{include});
            \item расширяется сценарием «Начисление штрафа за просрочку», если книга возвращена с опозданием (\textit{extend}).
        \end{itemize}
    \item \textbf{Добавление книги в каталог} — регистрация новой книги в системе;
    \item \textbf{Формирование отчёта по книгам} — подготовка отчёта о состоянии фонда, включающего просмотр задолженностей читателей (\textit{include}).
\end{itemize}

\section{Администратор}

Действующее лицо \textbf{Администратор} наследует все функции библиотекаря, а также имеет дополнительные возможности:
\begin{itemize}[noitemsep]
    \item \textbf{Удаление книги из каталога} — удаление устаревших или потерянных книг;
    \item \textbf{Просмотр отчётов} — просмотр сформированных библиотекарем отчётов;
    \item \textbf{Мониторинг состояния системы} — контроль корректной работы библиотечной системы и технических показателей.
\end{itemize}

\section{Связи include и extend}

\textbf{Include} используется для обозначения обязательных зависимостей между вариантами использования. Например:
\begin{itemize}
    \item «Проверка задолженности» включается в «Возврат книги»;
    \item «Проверка наличия книги» включается в «Выдачу книги».
\end{itemize}

\textbf{Extend} используется для описания альтернативных сценариев:
\begin{itemize}
    \item «Начисление штрафа за просрочку» расширяет «Возврат книги», если срок возврата нарушен;
    \item «Резервирование книги» расширяет «Выдачу книги», если книга недоступна.
\end{itemize}

\section{Заключение}

Диаграмма вариантов использования отражает взаимодействие основных пользователей с системой библиотеки. 
Она демонстрирует основные функции системы и взаимосвязи между ними, а также распределение ролей между действующими лицами.

\end{document}
